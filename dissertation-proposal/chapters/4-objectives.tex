\chapter{Objetivos}

\section{Objetivo Geral}

Desenvolver e avaliar modelos de aprendizado profundo para automatizar a detecção de EPBs em imagens airglow 630,0 nm do programa EMBRACE/INPE, e, com isso, contribuir para o aprimoramento do monitoramento operacional do clima espacial no Brasil.

\section{Objetivos específicos}

\begin{enumerate}
\item Realizar uma revisão sistemática da literatura sobre aplicações de machine learning e deep learning na detecção de irregularidades ionosféricas e estruturas em imagens airglow, identificando arquiteturas, técnicas de pré-processamento e métricas de avaliação utilizadas.

\item Construir um banco de dados rotulado de imagens airglow contendo exemplos de presença e ausência de EPBs, incluindo diferentes condições observacionais, estações do ano e níveis de atividade solar, a partir dos arquivos do programa EMBRACE/INPE.

\item Implementar e treinar diferentes arquiteturas de redes neurais convolucionais (U-Net, ResNet, EfficientNet, e variantes) para segmentação e detecção de bolhas de plasma em imagens airglow, explorando estratégias de transferência de aprendizado e aumento de dados.

\item Avaliar comparativamente o desempenho dos modelos desenvolvidos utilizando métricas quantitativas (precision, recall, F1-score, IoU) e validação com especialistas do domínio para identificar a arquitetura mais adequada ao problema.
\end{enumerate}