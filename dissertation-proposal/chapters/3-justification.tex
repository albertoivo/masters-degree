\chapter{Justificativa}

A relevância científica e tecnológica deste projeto se justifica pela necessidade de superar as limitações dos métodos tradicionais manuais e mitigar os impactos severos desses fenômenos nos sistemas tecnológicos modernos \cite{Githio2024}.

Como as EPBs apresentam morfologias altamente variáveis (estruturas ramificadas bifurcadas ou formas de "Y"~ou "C") \cite{Githio2024}, a detecção e segmentação automática, realizada por \textit{deep learning} em grandes volumes de dados, permite caracterizar e estudar essas estruturas finas e complexas de irregularidades ionosféricas com uma precisão e objetividade que são impossíveis de alcançar através da inspeção visual manual \cite{Zhong2025}.

Modelos automatizados permitem detecção e classificação em bem menos tempo. Em testes com imagens airglow do EMBRACE/INPE (2458 imagens de treinamento em hardware, embora não especificado mas com objetivo de ser de "baixo custo"), modelos baseados em 2DPCA e técnicas de Inteligência Artificial Explicável (XAI) apresentaram tempos de treinamento entre 2,18 e 8,71 minutos, contra 38,72–255,96 minutos de arquiteturas padrão de CNN. Essa eficiência não só reduz custos computacionais como também permite implementações em estações remotas com recursos limitados, acelerando a entrega de alertas de interferência em sistemas GNSS e melhorando a mitigação de impactos operacionais \cite{Yacoub2025}.