\chapter{Metodologia}

Esta pesquisa é caracterizada como um estudo experimental quantitativo, focado no desenvolvimento e avaliação de modelos de aprendizado profundo para a detecção automática de EPBs em imagens airglow 630,0 nm do programa EMBRACE/INPE. A metodologia proposta compreende as seguintes etapas principais:

\section{Revisão Sistemática da Literatura}

Uma revisão sistemática da literatura será conduzida para identificar e analisar trabalhos relevantes que aplicaram técnicas de machine learning e deep learning na detecção de irregularidades ionosféricas e estruturas em imagens airglow. Serão exploradas bases de dados acadêmicas como IEEE Xplore, Scopus, Web of Science e Google Scholar, utilizando palavras-chave relacionadas ao tema. A revisão buscará identificar arquiteturas de redes neurais, técnicas de pré-processamento, estratégias de aumento de dados e métricas de avaliação utilizadas em estudos anteriores.

\section{Construção do Banco de Dados Rotulado}

A construção do banco de dados rotulado será realizada a partir da seleção e anotação de imagens airglow 630,0 nm do programa EMBRACE/INPE. As imagens serão pré-processadas para garantir a qualidade e a consistência dos dados. Em seguida, será realizada a anotação manual das EPBs por especialistas, utilizando ferramentas de anotação de imagens. O banco de dados resultante será dividido em conjuntos de treinamento, validação e teste, seguindo as melhores práticas para garantir a generalização dos modelos desenvolvidos.

\section{Implementação e Treinamento de Modelos de Deep Learning}

Nesta etapa, serão implementados e treinados modelos de deep learning para a detecção de EPBs nas imagens airglow. Serão exploradas diferentes arquiteturas de redes neurais, adaptadas para a tarefa de segmentação de imagens. O treinamento será realizado utilizando o conjunto de dados rotulado, aplicando técnicas de aumento de dados para melhorar a robustez dos modelos. As métricas de avaliação, como IoU e F1-score, serão utilizadas para monitorar o desempenho dos modelos durante o treinamento e validação.