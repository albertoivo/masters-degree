\documentclass[
   12pt,
   oneside,
   a4paper,
   brazil
]{abntex2}

% Codificação e fontes (para pdflatex)
\usepackage[utf8]{inputenc}
\usepackage[T1]{fontenc}
\usepackage{lmodern}

% Microtipografia
\usepackage{microtype}

% Língua já vem de 'brazil' na classe, mas se quiser múltiplos idiomas:
% \usepackage[english,brazil]{babel}

% Figuras e gráficos
\usepackage{graphicx}
\graphicspath{{figuras/}}

% Matemática (se necessário)
%\usepackage{amsmath, amssymb, mathtools}

% Tabelas melhoradas
\usepackage{booktabs}
\usepackage{array}

% Listas
\usepackage{enumitem}
\setlist{nosep}

% Espaçamento inicial
\usepackage{indentfirst}

% Citações ABNT
\usepackage{abntex2cite}
\bibliographystyle{abntex2-alf}

% Hiperlinks
\usepackage{hyperref}
\hypersetup{
  pdftitle={DETECÇÃO AUTOMÁTICA DE BOLHAS DE PLASMA EQUATORIAL EM IMAGENS AIRGLOW UTILIZANDO TÉCNICAS DE APRENDIZADO PROFUNDO: UMA APLICAÇÃO AO PROJETO EMBRACE/INPE},
  pdfauthor={Alberto Ivo da Costa Vieira},
  pdfsubject={Pré-projeto de dissertação},
  pdfkeywords={Bolhas de Plasma Equatorial, Inteligência Artificial, Deep Learning, Machine Learning, Clima Espacial, EMBRACE, INPE},
  colorlinks=true,
  linkcolor=black,
  citecolor=black,
  urlcolor=blue
}

% Metadados/capa
\titulo{DETECÇÃO AUTOMÁTICA DE BOLHAS DE PLASMA EQUATORIAL EM IMAGENS AIRGLOW UTILIZANDO TÉCNICAS DE APRENDIZADO PROFUNDO: UMA APLICAÇÃO AO PROJETO EMBRACE/INPE}
\autor{Alberto Ivo da Costa Vieira}
\local{João Pessoa-PB}
\data{2025}
\orientador{Prof.\ Dr.\ Tiago Maritan Ugulino de Araujo}
\coorientador{Prof.\ Dr.\ Cosme Alexandre Oliveira Barros Figueiredo}
\instituicao{
  Universidade Federal da Paraíba
  \par
  Programa de Pós-Graduação em Informática
}
\tipotrabalho{Pré-projeto de dissertação de mestrado}

\begin{document}

% Pré-textuais essenciais
\imprimircapa
\imprimirfolhaderosto*

% --- Listas (somente Figuras e Tabelas) ---
% \listoffigures
\listoftables

% siglas
\begin{siglas}
  \item[2DPCA] Análise de Componentes Principais Bidimensional (Two-Dimensional PCA)
  \item[CNN] Rede Neural Convolucional (Convolutional Neural Network)
  \item[DL] Aprendizado Profundo (Deep Learning)
  \item[EMBRACE] Programa de Estudo e Monitoramento Brasileiro de Clima Espacial
  \item[EPB] Bolhas de Plasma Equatorial (Equatorial Plasma Bubble)
  \item[GNSS] Sistemas Globais de Navegação por Satélite (Global Navigation Satellite Systems)
  \item[INPE] Instituto Nacional de Pesquisas Espaciais
  \item[IoU] Intersection over Union
  \item[ML] Aprendizado de Máquina (Machine Learning)
  \item[SVM] Máquina de Vetor de Suporte (Support Vector Machine)
  \item[XAI] Inteligência Artificial Explicável (Explainable Artificial Intelligence)
\end{siglas}

% Sumário
\tableofcontents

% --- Capítulos (cada um em arquivo separado) ---
\chapter{Introdução}

As Bolhas de Plasma Equatorial (\textit{Equatorial Plasma Bubbles} - EPB) são zonas caracterizadas por flutuações na densidade do plasma que se formam na ionosfera de baixa latitude, principalmente durante o período após o pôr do sol. Elas submetem os sinais de rádio a variabilidades de amplitude e de fase, afetando o funcionamento de sistemas tecnológicos que utilizam os sinais dos Sistemas Globais de Navegação por Satélite (GNSS) para navegação \cite{Githio2024}.

A relevância do Brasil para a detecção, monitoramento e estudo das EPBs é extremamente alta, devido à sua localização geográfica e geomagnética única e ao papel ativo de suas instituições de pesquisa, como o Instituto Nacional de Pesquisas Espaciais (INPE), através do Programa de Estudo e Monitoramento Brasileiro de Clima Espacial (EMBRACE) cujo objetivo principal é monitorar, modelar e difundir informações sobre o Clima Espacial para prever e mitigar seus impactos sobre as atividades tecnológicas, econômicas e sociais no Brasil \cite{EMBRACE2021}.

Tradicionalmente, os eventos de EPBs são identificados principalmente pelo olho humano, um método demorado e ineficiente, facilmente influenciado pela subjetividade do observador e não é escalável para o volume de dados observacionais gerados atualmente. Por exemplo, um imageador airglow de canal único de uma única estação no Projeto Meridiano Chinês gera cerca de 1 TB de dados por ano, tornando o processamento manual um desafio significativo \cite{Zhong2025}.

Técnicas de aprendizado de máquina têm sido aplicadas para automação da detecção de EPBs. \citeonline{Thanakulketsarat2023} desenvolveram um modelo combinado de CNN/SVM com kernel RBF, alcançando acurácia de 93,67\%. \citeonline{Yacoub2025} propuseram uma abordagem utilizando Análise de Componentes Principais Bidimensional (2DPCA) para extração de características espaciais, combinada com \textit{Recursive Feature Elimination} (RFE) e classificador Random Forest, atingindo acurácia de 98,17\%. Embora promissores, estes estudos focaram principalmente em datasets específicos, havendo necessidade de adaptação às características particulares das imagens airglow brasileiras do programa EMBRACE/INPE.


Este projeto propõe o desenvolvimento de uma abordagem baseada em \textit{deep learning} para detecção automática de EPBs em imagens airglow do EMBRACE, explorando arquiteturas modernas de CNN adaptadas às características específicas dessas estruturas ionosféricas. A pesquisa situa-se na interseção entre ciências espaciais e inteligência artificial, contribuindo tanto para o avanço metodológico em visão computacional quanto para o aprimoramento dos sistemas operacionais de monitoramento de clima espacial no Brasil. Espera-se reduzir o tempo de detecção, aumentar a consistência histórica e fornecer base metodológica para aplicações preditivas subsequentes de clima espacial no Brasil.

\chapter{Problema de Pesquisa}

O monitoramento contínuo das bolhas de plasma equatorial através de imageadores airglow gera um volume substancial de dados diários que demandam processamento e análise sistemática. Cada estação do programa EMBRACE/INPE produz uma imagem a cada poucos minutos durante toda a noite, resultando em centenas de imagens por noite e dezenas de milhares de imagens por ano \cite{Githio2024}.

As EPBs possuem uma alta variabilidade morfológica e dinâmica, o que dificulta sua detecção automática. Elas podem variar em tamanho, forma e intensidade, e podem se mover rapidamente através do campo de visão dos imageadores airglow~\cite{Githio2024}. Além disso, as imagens frequentemente contêm ruído de fundo, nuvens, estrelas e poluição luminosa~\cite{Siddiqui2025} o que pode levar a falsos positivos ou negativos na detecção das EPBs.

Trabalhos recentes têm demonstrado a aplicabilidade de técnicas de deep learning para classificação e segmentação de estruturas em imagens airglow, porém lacunas importantes permanecem em aberto \cite{Githio2024}: (i) não existe um modelo otimizado especificamente para o contexto das imagens EMBRACE/INPE com suas características instrumentais particulares; (ii) a performance de diferentes arquiteturas de CNN (ResNet, U-Net, EfficientNet) não foi sistematicamente comparada para este problema específico; (iii) estratégias de aumento de dados (\textit{data augmentation}) e transferência de aprendizado (\textit{transfer learning}) carecem de investigação aprofundada neste domínio; e (iv) a integração de modelos de detecção em sistemas operacionais de tempo real ainda representa um desafio técnico.

Diante desse contexto, este projeto busca responder às seguintes questões de pesquisa:

\begin{enumerate}
\item Qual arquitetura de rede neural convolucional apresenta melhor desempenho na detecção automática de bolhas de plasma equatorial em imagens airglow 630,0 nm do programa EMBRACE/INPE?
\item Que estratégias de pré-processamento, aumento de dados e treinamento otimizam a capacidade de generalização dos modelos para diferentes condições observacionais e estações do ano?
\item Como métricas de classificação (\textit{accuracy, precision, recall, F1-score}) e métricas de segmentação morfológica em nível de pixel (\textit{Intersection over Union - IoU, Dice coefficient, pixel accuracy}) se complementam na avaliação da qualidade da detecção de EPBs?
\item É viável implementar um sistema de detecção automática em tempo quase-real integrado ao fluxo operacional do programa EMBRACE?
\end{enumerate}
\chapter{Justificativa}

A relevância científica e tecnológica deste projeto se justifica pela necessidade de superar as limitações dos métodos tradicionais manuais e mitigar os impactos severos desses fenômenos nos sistemas tecnológicos modernos \cite{Githio2024}.

Como as EPBs apresentam morfologias altamente variáveis (estruturas ramificadas bifurcadas ou formas de "Y"~ou "C") \cite{Githio2024}, a detecção e segmentação automática, realizada por \textit{deep learning} em grandes volumes de dados, permite caracterizar e estudar essas estruturas finas e complexas de irregularidades ionosféricas com uma precisão e objetividade que são impossíveis de alcançar através da inspeção visual manual \cite{Zhong2025}.

Modelos automatizados permitem detecção e classificação em bem menos tempo. Em testes com imagens airglow do EMBRACE/INPE (2458 imagens de treinamento em hardware, embora não especificado mas com objetivo de ser de "baixo custo"), modelos baseados em 2DPCA e técnicas de Inteligência Artificial Explicável (XAI) apresentaram tempos de treinamento entre 2,18 e 8,71 minutos, contra 38,72–255,96 minutos de arquiteturas padrão de CNN. Essa eficiência não só reduz custos computacionais como também permite implementações em estações remotas com recursos limitados, acelerando a entrega de alertas de interferência em sistemas GNSS e melhorando a mitigação de impactos operacionais \cite{Yacoub2025}.
\chapter{Objetivos}

\section{Objetivo Geral}

Desenvolver e avaliar modelos de aprendizado profundo para automatizar a detecção de EPBs em imagens airglow 630,0 nm do programa EMBRACE/INPE, e, com isso, contribuir para o aprimoramento do monitoramento operacional do clima espacial no Brasil.

\section{Objetivos específicos}

\begin{enumerate}
\item Realizar uma revisão sistemática da literatura sobre aplicações de machine learning e deep learning na detecção de irregularidades ionosféricas e estruturas em imagens airglow, identificando arquiteturas, técnicas de pré-processamento e métricas de avaliação utilizadas.

\item Construir um banco de dados rotulado de imagens airglow contendo exemplos de presença e ausência de EPBs, incluindo diferentes condições observacionais, estações do ano e níveis de atividade solar, a partir dos arquivos do programa EMBRACE/INPE.

\item Implementar e treinar diferentes arquiteturas de redes neurais convolucionais (U-Net, ResNet, EfficientNet, e variantes) para segmentação e detecção de bolhas de plasma em imagens airglow, explorando estratégias de transferência de aprendizado e aumento de dados.

\item Avaliar comparativamente o desempenho dos modelos desenvolvidos utilizando métricas quantitativas (precision, recall, F1-score, IoU) e validação com especialistas do domínio para identificar a arquitetura mais adequada ao problema.
\end{enumerate}
\chapter{Metodologia}

Esta pesquisa é caracterizada como um estudo experimental quantitativo, focado no desenvolvimento e avaliação de modelos de aprendizado profundo para a detecção automática de EPBs em imagens airglow 630,0 nm do programa EMBRACE/INPE. A metodologia proposta compreende as seguintes etapas principais:

\section{Revisão Sistemática da Literatura}

Uma revisão sistemática da literatura será conduzida para identificar e analisar trabalhos relevantes que aplicaram técnicas de machine learning e deep learning na detecção de irregularidades ionosféricas e estruturas em imagens airglow. Serão exploradas bases de dados acadêmicas como IEEE Xplore, Scopus, Web of Science e Google Scholar, utilizando palavras-chave relacionadas ao tema. A revisão buscará identificar arquiteturas de redes neurais, técnicas de pré-processamento, estratégias de aumento de dados e métricas de avaliação utilizadas em estudos anteriores.

\section{Construção do Banco de Dados Rotulado}

A construção do banco de dados rotulado será realizada a partir da seleção e anotação de imagens airglow 630,0 nm do programa EMBRACE/INPE. As imagens serão pré-processadas para garantir a qualidade e a consistência dos dados. Em seguida, será realizada a anotação manual das EPBs por especialistas, utilizando ferramentas de anotação de imagens. O banco de dados resultante será dividido em conjuntos de treinamento, validação e teste, seguindo as melhores práticas para garantir a generalização dos modelos desenvolvidos.

\section{Implementação e Treinamento de Modelos de Deep Learning}

Nesta etapa, serão implementados e treinados modelos de deep learning para a detecção de EPBs nas imagens airglow. Serão exploradas diferentes arquiteturas de redes neurais, adaptadas para a tarefa de segmentação de imagens. O treinamento será realizado utilizando o conjunto de dados rotulado, aplicando técnicas de aumento de dados para melhorar a robustez dos modelos. As métricas de avaliação, como IoU e F1-score, serão utilizadas para monitorar o desempenho dos modelos durante o treinamento e validação.
\chapter{Cronograma}

\begin{table}[htb]
    \centering
    \caption{Cronograma de Execução da Dissertação (8 Trimestres)}
    \label{tab:cronograma}
    \begin{tabular}{p{5.5cm} *{8}{c}}
        \toprule
        \textbf{Atividades} & \textbf{T1} & \textbf{T2} & \textbf{T3} & \textbf{T4} & \textbf{T5} & \textbf{T6} & \textbf{T7} & \textbf{T8} \\
        \midrule
        \textbf{Revisão Sistemática da Literatura} & X & X &  &  &  &  &  &  \\
        \addlinespace
        \textbf{Construção do Banco de Dados Rotulado} &  & X & X &  &  &  &  &  \\
        \addlinespace
        \textbf{Pré-processamento e Data Augmentation} &  &  & X & X &  &  &  &  \\
        \addlinespace
        \textbf{Implementação de Modelos de DL} &  &  & X & X & X &  &  &  \\
        \addlinespace
        \textbf{Experimentos de Transfer Learning e XAI} &  &  &  & X & X &  &  &  \\
        \addlinespace
        \textbf{Avaliação Quantitativa e Qualitativa (IoU, F1, especialistas)} &  &  &  &  & X & X &  &  \\
        \addlinespace
        \midrule
        \textbf{Redação dos Capítulos de Metodologia e Resultados} &  &  &  & X & X & X &  &  \\
        \addlinespace
        \textbf{Análise Crítica e Discussão dos Resultados} &  &  &  &  & X & X & X &  \\
        \addlinespace
        \textbf{Preparação para Qualificação (Pré-banca)} &  &  &  &  &  & X &  &  \\
        \addlinespace
        \textbf{Revisão e Ajustes Pós-Qualificação} &  &  &  &  &  &  & X &  \\
        \addlinespace
        \textbf{Submissão de Artigos para Conferência/Revista} &  &  &  &  &  & X & X &  \\
        \addlinespace
        \textbf{Redação Final da Dissertação e Formatação ABNT} &  &  &  &  &  &  & X & X \\
        \addlinespace
        \textbf{Defesa Final da Dissertação} &  &  &  &  &  &  &  & X \\
        \bottomrule
    \end{tabular}
\end{table}


% Referências
\bibliography{references}

\end{document}
